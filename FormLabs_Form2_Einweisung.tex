% !TeX spellcheck = de_DE_frami
%%%%%%%%%%%%%%%%%%%%%%%%%%%%%%%%%%%%%%%%%%%%%%%%
% COPYRIGHT: (C) 2012-2015 FAU FabLab and others
% Bearbeitungen ab 2015-02-20 fallen unter CC-BY-SA 3.0
% Sobald alle Mitautoren zugestimmt haben, steht die komplette Datei unter CC-BY-SA 3.0. Bis dahin ist der Lizenzstatus aller alten Bestandteile ungeklärt.
%%%%%%%%%%%%%%%%%%%%%%%%%%%%%%%%%%%%%%%%%%%%%%%%


\newcommand{\basedir}{fablab-document}
\documentclass{\basedir/fablab-document}
\usepackage{amssymb} % Symbole für Knöpfe
\usepackage{subfigure,caption}
\usepackage{eurosym}
\usepackage{textcomp} % \textcelsius
\usepackage{tabularx} % Tabellen mit bestimmtem Breitenverhältnis der Spalten
\usepackage{wrapfig} % Textumlauf um Bilder
\usepackage{float} % Ermöglicht H als Platzierungsoption
\usepackage{array}
\usepackage[hyphens]{url}

\renewcommand{\texteuro}{\euro}
\newcommand{\fachbegriff}[1]{(\textit{#1})}
\newcommand{\ts}[1]{\textsuperscript{#1}}
\newcommand{\ra}{$\Rightarrow$}

% \linespread{1.2}
% \fancyhead{}
\date{\today}
\author{kontakt@fablab.fau.de}
\title{Einweisung SLA 3D-Drucker}

\begin{document}

\maketitle
\begin{center}
    Für LCD basierte SLA 3D-Drucker
\end{center}

\textbf{Nur eingewiesene Benutzer dürfen den Drucker selbständig bedienen}, um Beschädigungen zu vermeiden. Wenn du noch nicht eingewiesen bist, wende dich an einen Betreuer. Er erklärt dir die Bedienung und lässt dich unter Aufsicht das gewünschte Teil drucken. Wenn du alles verstanden hast, kannst auch du die Einweisung unterschreiben und den Drucker in Zukunft selbstständig benutzen.

\renewcommand{\contentsname}{Inhaltsverzeichnis / Arbeitsablauf}
\setcounter{tocdepth}{2}
\tableofcontents
\newpage

\section{Regeln und Hinweise}
% Hier sollte nur das wichtigste gegen "Kaputtmachen" des Druckers stehen.
Für die Benutzung ist es wichtig, dass du folgende Hinweise beachtest:

\begin{itemize}
    \item Obwohl der 3D-Druck unter den Begriff \enquote{Rapid Prototyping} fällt, kann ein gedrucktes Teil je nach Abmessungen durchaus mehrere Stunden in Anspruch nehmen. Betreuer und Software helfen dabei, die Dauer abzuschätzen.
    \item Wenn du nicht bis zum Ende deines Drucks da sein kannst, frage vorher einen Betreuer und hinterlasse einen Zettel mit deinem Namen und deinen Kontaktdaten.
    \item Anleitung ist exakt zu beachten. Wenn du nicht weiter weißt oder dir unsicher bist, frag einen Betreuer.
    \item Ein Materialwechsel und die Wartung darf \textbf{nur von einem Betreuer} durchgeführt werden.
    \item \textbf{Bezahlen bitte nicht vergessen!}
\end{itemize}
%\newpage
\subsection{Gefahren für Mensch und Maschine}
\begin{table}[h]
    \centering
    \begin{tabular}{ccc}

        \includegraphics[width=2cm]{bilder/GHSa.png}  & \includegraphics[width=2cm]{bilder/GHSf.png}
    \end{tabular}
\end{table}

\begin{itemize}
    \item Das eingesetzte Kunstharz und Isopropylalkohol sind gesundheitsschädlich. Haut- und Augenkontakt sowie Verschlucken und Einatmen der Dämpfe vermeiden.
    \item Zu Ihrer eigenen Sicherheit sollten sowohl beim Umgang mit dem flüssigen Kunstharz als auch beim Umgang mit Isopropylalkohol zur Nachbehandlung des Druckteils Nitrilhandschuhe getragen werden.
    \item Isopropylalkohol ist leicht entzündlich, daher ist der Umgang mit offenem Feuer und Zündquellen in der Nähe verboten.
    \item Drucke sind \textbf{nicht lebensmittelecht} oder biokompatibel, da sie auch nach der Weiterverarbeitung noch Spuren des Ausgangsmaterials enthalten können. 
\end{itemize}

%%%%%%%%%%%

\subsection{Persönliche Schutzausrüstung}

\begin{table}[h]
    \centering
    \begin{tabular}{cc}

        \includegraphics[width=2cm]{bilder/gaugenschutz.png}  & \includegraphics[width=2cm]{bilder/ghandschuh.png} \\
    \end{tabular}
\end{table}

Bei allen Arbeiten am Drucker und bei der Nachbearbeitung der Teile sind zu tragen:
\begin{itemize}
    \item \textbf{Schutzbrille}
    \item \textbf{Einweghandschuhe}
\end{itemize}
Erst nach dem Waschen und Nachbelichten dürfen die Teile ohne Handschuhe angefasst werden. Vorher können sich noch Reste von unpolymerisiertem Kunststoff auf der Oberfläche befinden.
\newpage

% % % % % % %

% % % % % % % % % % % % % % % %
\section{3D-Modell erstellen}

\subsection{Dateiformat}

Im STL-Dateiformat, Einheit: Millimeter. Alle gängigen 3D-Programme haben einen STL-Export.

\begin{itemize}
    \item auf \href{https://thingiverse.com}{Thingiverse.com} oder  \href{https://printables.com}{Printables.com}  gibt es viele vorgefertigte Modelle, als
        Grundlage oder gleich zum Bereit zum ausdrucken.
    \item oder erstelle ein Modell mit einem Programm deiner Wahl
        \begin{table}[H]
            \centering
            \begin{tabularx}{\textwidth}{|l|X|}
                \hline \textbf{Name} & \textbf{Beschreibung} \\
                \hline \multicolumn{2}{|c|}{\textit{kostenlose Software}}  \\
                \hline Blender & relativ komplex aber auch für Freiformflächen geeignet  \\
                \hline OpenSCAD & Skriptsprache für Konstruktion aus geometrischen Grundkörpern \\
                \hline DesignSpark Mechanical & Angelehnt an professionelle CAD-Software, aber relativ einfach zu bedienen  \\
                \hline TinkerCAD & sehr einfach, für Kinder gut geeignet  \\
                \hline Google SketchUp & wenig Einarbeitung, geringer Funktionsumfang, für einfache Teile \\
                \hline & \\
                \hline \multicolumn{2}{|c|}{\textit{kostenpflichtige Software (proprietär)}}  \\
                \hline PTC Creo, Solid Edge, Siemens NX & kostenlos beim RRZE für Studenten, professionelle Software \\
                \hline Autocad Inventor, Fusion 360 & kostenlos bei Autodesk für Studenten ebenfalls für professionelle Anwendungen \\
                \hline
            \end{tabularx}
        \end{table}
\end{itemize}

\subsection{Einschränkungen der Formen}
\begin{itemize}%228*128*H245mm
    \item \textbf{Bauraum}\\L\,$\times$\,B\,$\times$\,H\@: 228\,mm\,$\times$\,128\,mm\,$\times$\,245\,mm
    \item \textbf{Stützstrukturen}\\Um die Druckteile sinnvoll drucken zu können, müssen Stützstrukturen verwendet werden. Die Software erzeugt dazu ein loses Geflecht von Verbindungen zu Überhängen und Brücken, die nach dem Drucken mit einer Zange oder einem Skalpell entfernt werden können.
    \item \textbf{Ablösewanne}\\Bei Verwendung von Stützstrukturen kann automatisch eine Ablösewanne unter dem Objekt erzeugt werden. Diese erleichtert das Ablösen von der Bauplattform. Am Rand der Ablösewanne wird zusätzlich der Dateiname der STL-Datei ausgedruckt.
\end{itemize}

\newpage

\subsection{Designrichtlinien für SLA Drucke}

Die folgenden Konstruktionsrichtlinien helfen bei der Optimierung des Druckteils für den SLA-Druck und beschreiben die empfohlenen Mindestwerte. Obwohl es nicht verboten ist, diese Werte zu über- oder unterschreiten, wird empfohlen, dies nur dann zu tun, wenn ein konkreter Grund dafür vorliegt, und auch dann nur mit Vorsicht. Unter Umständen erfordert ein fehlgeschlagener Druck einen erhöhten Aufwand bei der nächsten Wartung. \\

\begin{itemize}
    \item \textbf{Mindestdicke von gestützten Wänden:}  0,4 mm\\
        \\
        \textbf{Hinweis:}Bei der Nachbehandlung von dünnen Wänden ist Vorsicht geboten, da diese Isopropanol aufnehmen und aufquellen können. Dies kann zu Verformungen führen, daher das Druckteil zum Waschen nur kurz eintauchen. \\
    \item \textbf{Mindestdicke nicht gestützter Wände:}  0,6 mm
    \item \textbf{Maximale Länge nicht gestützter Überhänge:}  1,0 mm
    \item \textbf{Mindestwinkel für nicht gestützte Überhänge:}  19$^\circ$ von der Ebene
    \item \textbf{Maximale horizontale Stützbrücke:}  21 mm
    \item \textbf{Mindestdurchmesser von vertikalen Drähten:}  1,5 mm
    \item \textbf{Mindestwert für geprägte Details:}  0,1 mm
    \item \textbf{Mindestwert für eingravierte Details:}  0,4 mm
    \item \textbf{Mindestabstand zwischen mehreren Druckteilen:}  1 mm
    \item \textbf{Mindestdurchmesser für Aussparungen:}  0,5 mm
    \item \textbf{Hohlkörper} \\
        Abflussmöglichkeiten für das verbleibende Kunstharz vorsehen, da dieses im Inneren verbleibt. Aufgrund der Viskosität des Harzes sollten diese Öffnungen einen Durchmesser von mindestens 3,5\,mm aufweisen.

\end{itemize}

\newpage

%%%%%%%%%%%%

\section{3D-Modell mit dem Slicer umwandeln und ausdrucken}

\subsection{Vorbereitung}

\begin{itemize}
    \item Vor dem Druckt ist es notwendig, die STL-Datei mit einem eindeutigen Namen zu versehen, zum Beispiel \enquote{Max\_Mustermann\_E-Mail\_Dateiname}.
    \item Programm \enquote{LyceeSlicer} öffnen und als Drucker den \enquote{Uniformation GKTwo}, auswählen.
    \item Mit Klick auf den Dialog \enquote{Datei} und anschließend mit \enquote{Öffnen} STL-Datei öffnen.
    \item Bei Bedarf kann man das Druckteil mit den Schaltflächen skalieren, drehen und verschieben.
    \item Notwendige Stützstrukturen automatisch erstellen lassen, bei Bedarf händisch anpassen. Hierzu die entsprechenden Menüpunkte oben mittig auswählen.
    \item Schichtdicke ist fix eingestellt auf 50\,µm. Änderungen davon geschehen auf Abspreche mit dem Maschinenverantwortlichen. 
\end{itemize}

\subsection{Drucken}
\begin{itemize}

    \item Zum Drucken des Objektes die orange Schaltfläche \enquote{Druckeinrichtung} öffnen.\\
    \item \textbf{Hinweis:} Notiere dir bitte die nun angezeigte kalkulierte Menge an Harz, die für dein Druckteil benötigt wird. Diese ist notwendig, um deinen Druck bezahlen zu können.
    \item In diesem Dialog ist vor dem Senden der Name des Druckteils zu deklarieren. Hier bitte mindesens Name und E-Mail-Adresse angeben, am besten auch noch die Telefonnummer.\\
    \item Nach erfolgreichem Senden befindet sich das Druckteil nun auf dem Drucker selbst. Hierzu einfach den Anweisungen auf dem Bildschirm des \textit{Form2} folgen bis der Druck gestartet ist.
    \item Wenn du nicht bis zum Ende deines Drucks da sein kannst, frage vorher einen Betreuer und hinterlasse einen Zettel mit deinem Namen und deinen Kontaktdaten.

\end{itemize}

\pagebreak

%%%%%%%%%%%%%%%%%%

\section{Nachbereitung}

Nach erfolgreichem Druck kann nun mit der Nachbearbeitung des Druckteils begonnen werden. Für diesen Zweck stehen nun zwei separate Gerätschaften zur Verfügung, nämlich der Form Wash und der Form Cure. 
Bei dem erstgenannten Gerät, dem Form Wash, handelt es sich um einen Isopropanol Behältnis mit eingebautem Rührfisch zur Reinigung der Druckteile von nicht verfestigten Herzrückständen. Der Form Cure hingegen ist eine optimierte UV-Aushärtekammer mit zusätzlicher Temperaturregelung. 

Die Nachbearbeitung ab dem fertigstellen des Druckvorgangs ist dabei wie folgt. 

\textbf{Hinweis:} Ab diesen Zeitpunkt muss die vorgeschriebene Schutzausrüstung (Handschuhe) getragen werden.

\begin{enumerate}
    \item Die ganze Konstruktionsplattform im Drucker lösen.
        \begin{itemize}
            \item Ein Hebel befindet sich dazu an der Oberseite der Konstruktionsplattform. Diesen nun lösen und die Quadratische Druckplattform zu sich ziehen und dabei herausnehmen.
	    \item Bitte dabei möglichst schonend mit der Kostruktionsplattform umgehen, denn es kann sein das sich Tropfen mit Harz darunter sammeln und den Arbeitsbereich verdrecken.
            \item Die grüne Haube des Druckers wieder vorsichtig verschließen, sonst kann es zudem zu Verunreinigungen des Harzes im Tank kommen.
        \end{itemize}
    \item Die Druckplatte aus dem Drucker nehmen, den Druck von dieser Plattform entfernen und in den Form Wash einlegen.
    \item Nun die Druckplatte reinigen und nach kurzer Trocknungszeit wieder in den Drucker auf die Haltevorrichtung einsetzen. 
    \item Form Wash nun starten und abwarten.
    \begin{itemize}
    	\item Die Standartzeit eines Waschdurchgangs beträgt 5 Minuten.
    	\item Zum Starten im Menü des Wash START mit dem Dreh-Drück-Taster auswählen und mit einem sanften Druck bestätigen.
    \end{itemize}
        	\item Nach Abschluss des Waschdurchgangs fährt die Druckplatform eigenständig aus dem Wash heraus.
        	\item Im besten Falle das Druckerzeugnis noch etwas hängen lassen, damit sich das Isopropanol vom Druck verflüchtigen kann.
        	\item Ist der Druck vom Wash entnommen, kann im Menü des Wash die Haltevorrichtung wieder eingefahren werden. Hierzu Sleep auswählen und bestätigen. .
            \item \textbf{Hinweis:} Bitte dabei möglichst schonend mit der Kostruktionsplattform umgehen, denn es kann sein das sich Tropfen mit Harz darunter sammeln und den Arbeitsbereich verdrecken.
    \item Nachhärtung in dem Form Cure
        \begin{itemize}
            \item Hierzu den Druck in den Cure einlegen.
            \item Das Material mindestens 15 min aushärten lassen. Hierzu im Menü des Cure die Parameter einstellen und bestätigen.
        \end{itemize}

    \item Entfernen der Stützstrukturen

\end{enumerate}

%%%%%%%%%%%%%%%%%

\section{Bezahlen und Abschließen}

Nach der Nachbearbeitung des Druckteils ist die Arbeitsfläche aufzuräumen und dem Nächsten den Platz frei zu machen, und, sofern noch nicht geschehen, das verbrauchte Harzvolumen am Drucker abzulesen und zu notieren.\\
Abgerechnet wird das Druckteil am Kassenterminal, hierzu ist das abgelesene Volumen Harz in $ml$ zu kennen, und am Kassenterminal auf ganze $ml$ aufgerundet zu bezahlen.

%%%%%%%%%%%%

\section{Pflege \& Wartung --- \textcolor{red}{Nur von Betreuer durchzuführen}}

\subsection{Überprüfung des Verschmutzungsgrades des Isopropanols im FormWash}
Da beim Waschen im FormWash nicht auspolymerisiertes Harz in das Isopropanol übergeht bzw. darin gelöst wird, ist Dieses nach einer gewissen Zeit verschmutzt und muss ausgetauscht werden. Da Isopropanol eine geringere molare Masse als das Harz hat, steigt mit zunehmender Harzkonzentration im Isopropanol die Dichte des Gemisches. Der Verschmutzungsgrad kann daher über eine simple Dichtemessung bestimmt werden.

Das eingesetzte Verfahren wird als Aräometrie bezeichnet und verwendet eine Tauchspindel mit einem definierten Gewicht und Volumen. Entsprechend dem archimedischen Prinzip verdrängt die Spindel jeweils ihr Eigengewicht, so dass sie in dichteren Flüssigkeiten weniger tief eintaucht. Die Eintauchtiefe ist daher linear proprotional zur Dichte der Flüssigkeit die gemessen wird.

Die Tauchspindel des FormWash befindet sich hinten rechts am FormWash hinter einer Klappe. An der Spindel sind auf dem Schwimmer kurze und lange "Flügel" nach oben. Nachdem frisches Isopropanol in den FormWash gegeben wurde, wird der rote Gummiring an der Spindel so eingestellt und positioniert, dass er auf der Höhe der kurzen "Flügel", also an der Oberkante des Körpers des Schwimmers ist. Damit ist der Nullpunkt festgelegt. Mit zunehmender Verschmutzung und damit Dichte schwimmt die Spindel mit dem roten Gummiring bei jeder Messung etwas höher. Sobald er bei einer Messung auf der Höhe der beiden längeren (oberen) Flügel schwimmt, ist das Isopropanol verbraucht und muss baldmöglichst ausgetauscht werden.

\subsection{Wechseln des Harzes}
Der SLA-Drucekr ist von uns dafür gedacht, dass das Harz gewechselt werden kann. Dies ist jedoch nur von einem eingewiesenen Betreuer durchzuführen. 

Hierzu muss sichergestellt werden, dass die Druckplattform gereinigt und frei von Kontaminationen ist. Auch soll darauf geachtet werden, dass keine Rückstände von Isopropanol auf der Druckplattform zu finden sind.\\
Anschließend kann der gesamte Harztank gelöst werden, dies indem man links und rechts am Tank greift und, wie zuvor, mit einem Sanften Zug zum Benutzer hin den Tank löst. \\

\subsection{nFEP Folie}
Zur Reinigung des der nFEP Folie unter dem Harztanks bei Verschmutzung durch Schlieren oder Fingerabdrücken.

%%%%%%%%%%%%

\ccLicense{FormLabs\_Form2\_Einweisung}{Einweisung FormLabs Form2}

\end{document}
